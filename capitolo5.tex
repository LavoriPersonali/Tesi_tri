%%%%%%%%%%%%%%%%%%%%%%%%%%%%%%%%%%%%%%%%%%%%%%%%%%%%%%%%%%%
% Capitolo 5

\chapter{Conclusioni e Lavori Futuri}
\label{ref:Conclusioni}

\section{Lavori futuri}
Come accennato in precedenza, la scelta stessa del linguaggio di programmazione(Python), per sviluppare il tool per l'analisi della rete locale, � stata fatta per rendere questo lavoro di Tesi come un punto di partenza, da estendere e migliorarne le capacit� di deduzione riguardo le reti di calcolatori e i dispositivi connessi ad esse.\\
Anche la considerevole quantit� di protocolli ed applicativi che fanno uso dei canali di Broadcast e Multicast � un'ottimo spunto dal quale potrebbe scaturire il proseguimento nello sviluppo di questo Elaborato.\\

Un esempio potrebbe essere quello di estendere il tool aggiungendo la capacit� di analizzare dati provenienti da protocolli come NetBIOS o \textit{Microsoft Windows Browser Protocol}, incrementando le informazioni acquisite, e quindi andare ad identificare tutti quei dispositivi che non usufruiscono dei protocolli gi� inclusi nello strumento di analisi, e che quindi non vengono \qts{rilevati} da quest'ultimo.\\

Un'altra idea per estendere questo lavoro � incrementare la capacit� di analisi del tool aggiungendo la possibilit� di riconoscere \textit{nomi} di persone, dispositivi, e quant'altro possa rivelare ulteriori informazioni a partire dalle stringhe reperite dagli hostnames dei nodi della rete. Un punto di partenza sarebbe aggiungere ulteriori dizionari di keyword da utilizzare per identificare la tipologia di dispositivo che le integra nel proprio hostname. O ancora, aggiungere la capacit� di reperire in rete, tramite appositi servizi o API, elenchi aggiornati con i nomi, cognomi e diminutivi di persone comunemente usati, e quindi riuscendo con maggiore facilit� e precisione ad attribuire un proprietario ad un dato dispositivo.\\

Una capacit� che per motivi di tempo non � stata sviluppata, � quella di dettagliare ulteriormente la caratterizzazione dei dispositivi andando ad analizzare i campi \textit{TXT} di tutti quei servizi che includono informazioni riguardo le caratteristiche Hardware e Software del dispositivo, non limitandosi a quelle dei record \textit{ \_device-info } annunciati solo dai dispositivi Apple.\\
I servizi che li diffondono sono numerosi, a partire dai protocolli di stampa, come \textit{\_ipp} che forniscono moltissimi dettagli sulla stampate di rete, fino ad arrivare a servizi che fungono da \textit{appoggio} a protocolli e applicativi proprietari, come ad esempio \textit{\_nomachine} che diffonde informazioni riguardo la macchina che si rende disponibile al controllo di desktop remoto, fornendo informazioni come: \textit{OS=Ubuntu 18.04.1 LTS} e \textit{uuid=\dots} .\\

Un'altra funzionalit� che potrebbe incrementare il numero di informazioni estratte dai campi mDNS � quella di effettuare delle \qts{query attive} per il discovery di tutti quei dispositivi che, s� offrono servizi, ma che magari a causa di comportamenti pi� conservativi non divulgano informazioni a riguardo se non direttamente interpellati. 
Con questa utilit� quindi si va a stimolare in maniera diretta il nodo in questione, incrementando quindi le informazioni riguardo i servizi offerti.\\

\section{Conclusioni}

Questo studio riguardo la raccolta di informazioni su reti locali e dispositivi ad esse connessi si � rivelato pi� soddisfacente, considerata la quantit� di informazioni a cui si � avuto accesso.\\

L'obiettivo di \qts{anonimato} e discrezione per quanto la raccolta dati, e lo strumento utilizzato a tale scopo, � stato pienamente raggiunto, in quanto non � possibile in alcun modo identificare il tool che in quel momento sta catturando ed analizzando i pacchetti che circolano nella rete.
Infatti, lo strumento si limita ad analizzare i pacchetti, e quindi le informazioni in essi contenuti, che vengono recapitati \textit{direttamente} alla macchina sulla quale � in esecuzione, non compiendo alcuna azione che potrebbe rendere visibile l'\textit{attivit� di analisi} della rete.\\

Per quanto riguarda il tipo di informazioni che � stato possibile ricavare dai dati reperiti dalle comunicazioni Broadcast e Multicast, si sono rivelati sorprendenti, svelando dettagli, specialmente per quanto riguarda i dispositivi, decisamente \qts{personali}, e che potrebbero essere sfruttati come punti di debolezza per eventuali attacchi da parte di intrusi nella rete, o utenti \qts{malintenzionati}.\\
Riguardo ai dispositivi, � stato possibile identificare, nei casi pi� fortuiti: modello del dispositivo, e quindi tutte le caratteristiche Hardware di cui dispone, e Sistema Operativo, nonch� versione installata in quel momento, fornendo dettagli del Software che vi � in esecuzione.
Inoltre, sempre riguardo i dispositivi, vengono fornite informazioni sulle periferiche che in quel momento sono a loro connesse, come ad esempio stampanti, scanner, ed altro materiale 
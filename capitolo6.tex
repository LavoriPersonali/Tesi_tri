%%%%%%%%%%%%%%%%%%%%%%%%%%%%%%%%%%%%%%%%%%%%%%%%%%%%%%%%%%%
% Capitolo 6

\chapter{Conclusioni}
\label{ref:Conclusioni}

Questo studio riguardo la raccolta di informazioni su reti locali e dispositivi ad esse connessi si � rivelato pi� soddisfacente, considerata la quantit� di informazioni a cui si � avuto accesso.\\

L'obiettivo di \qts{anonimato} e discrezione per quanto la raccolta dati, e lo strumento utilizzato a tale scopo, � stato pienamente raggiunto, in quanto non � possibile in alcun modo identificare il tool che in quel momento sta catturando ed analizzando i pacchetti che circolano nella rete.
Infatti, lo strumento si limita ad analizzare i pacchetti, e quindi le informazioni in essi contenuti, che vengono recapitati \textit{direttamente} alla macchina sulla quale � in esecuzione, non compiendo alcuna azione che potrebbe rendere visibile l'\textit{attivit� di analisi} della rete.\\

Per quanto riguarda il tipo di informazioni che � stato possibile ricavare dai dati reperiti dalle comunicazioni Broadcast e Multicast, si sono rivelati sorprendenti, svelando dettagli, specialmente per quanto riguarda i dispositivi, decisamente \qts{personali}, e che potrebbero essere sfruttati come punti di debolezza per eventuali attacchi da parte di intrusi nella rete, o utenti \qts{malintenzionati}.\\
Riguardo ai dispositivi, � stato possibile identificare, nei casi pi� fortuiti: modello del dispositivo, e quindi tutte le caratteristiche Hardware di cui dispone, e Sistema Operativo, nonch� versione installata in quel momento, fornendo dettagli del Software che vi � in esecuzione.
Inoltre, sempre riguardo i dispositivi, vengono fornite informazioni sulle periferiche che in quel momento sono a loro connesse, come ad esempio stampanti, scanner, ed altro materiale \qts{da ufficio}.
In casi pi� generici invece, vene comunque identificata correttamente la \textit{Classe} a cui il dispositivo appartiene, grazie ai servizi applicativi che esso offre, fornendo un'idea generica di come vengano sfruttate le reti locali analizzate: per utilizzo \qts{d'ufficio} se vengono rilevate principalmente stampanti, fax e scanner, o magari per il solo accesso alla rete Internet globale, se rilevati principalmente dispositivi mobili, o ancora per un utilizzo \qts{privato} e condivisione di \textit{Media}, se vengono rilevati numerosi dispositivi atti a condividere file, musica, streaming video e audio, e quant'altro di simile.\\
Tutte queste informazioni riguardo i dettagli dei dispositivi, potrebbero essere sfruttate per \textit{indagini di mercato} da tutte quelle aziende che lavorano nel settore, e quindi sfruttando le \qts{abitudini} degli utenti di una determinata rete locale, potrebbero ricavare informazioni utili ai loro scopi, come intraprendere o meno determinate scelte riguardo la vendita e promozione dei propri prodotti.\\

Per quanto riguarda gli utenti, le persone fisiche che quindi in alcuni casi possiedono il dispositivo, l'informazione pi� personale che si riesce a ricavare da questo studio sono il \textit{nome} e \textit{cognome}.
Per quanto possa a prima vista non sembrare particolarmente rilevante, sono comunque informazioni che appartengono alla sfera privata degli utenti, e quindi visto il modo con cui vengono diffuse, andrebbero trattate con pi� attenzione.
Questo offre un ottimo spunto di riflessione su come i nostri dati personali vengono utilizzati, e magari dare maggiore consapevolezza dei rischi che si corre utilizzando determinati dispositivi, servizi o tecnologie che a prima vista risultano estremamente comode, ma che tutto sommato comportano degli svantaggi e un prezzo da pagare per quella comodit�: le proprie informazioni.\\

Sempre restando nella \textit{sfera privata} degli utenti, le informazioni ricavate dai pacchetti Dropbox sono ancor pi� rilevanti.
Di fatto, non solo svelano dettagli delicati riguardo la topologia della rete, ma bens� rivelano anche informazioni inerenti alle interazioni sociali che si hanno fra gli utilizzatori di tali reti locali.\\
Queste informazioni possono dare inizio a numerosissime supposizioni a riguardo: a partire dal dedurre quali sono i gruppi di lavoro con i quali vengono organizzate le attivit� di un'azienda, o ancora i ruoli che ricoprono gli utenti che fanno parte di una determinata organizzazione; pi� semplicemente, possono rivelare le relazioni private che incorrono fra i possessori dei dispositivi, ad esempio: se 2 utenti di 2 dispositivi condividono una cartella o dei file, molto probabilmente hanno una qualche sorta di interazione, o per lo meno si conoscono.\\

Ma tutte le informazioni scaturite da questo studio, come possono essere utilizzate? chi potrebbe essere interessato a sfruttarle?\\
Oltre che ad utilizzi dannosi o con il fine di \qts{invadere} la sfera privata degli utenti per scopi di Marketing, questo � un prezioso strumento per conoscere la propria rete locale, e quindi attuare azioni atte a contrastare queste intrusioni.
In primo luogo potrebbero essere usate per incrementare il livello di sicurezza, magari impedendo completamente la circolazione di tali informazioni, o quanto meno restringerle a determinati nodi, suddividendo la rete locale in sotto-reti autonome, e quindi relegandone il traffico all'interno, dato che le comunicazioni Broadcast e Multicast sono limitate alla rete locale di cui fanno parte.\\
Un'altro utilizzo sarebbe quello di monitoraggio di una rete locale, tenendo traccia di quali dispositivi vengono connessi, quali utenti ne fanno parte, e magari rilevando accessi indesiderati da parte di utenti non autorizzati.\\
In fine, pi� semplicemente, queste informazioni potrebbero rendere consapevoli gli utenti di quali dati rendono disponibili, e quindi dare la possibilit� di scegliere le azioni da compiere a riguardo, da un semplice cambio di \textit{hostname}, a magari interrompere l'utilizzo di determinati servizi o tecnologie in favore di altre che offrono gli stessi vantaggi con una maggiore attenzione riguardo i dati diffusi in rete.
%%%%%%%%%%%%%%%%%%%%%%%%%%%%%%%%%%%%%%%%%%%%%%%%%%%%%%%%%%%
% Capitolo 1

\chapter{Introduzione}
\label{ref:Introduzione}

Dispositivi e Applicazioni sono sempre pi� interconnessi tra loro, comunicando e scambiando informazioni, condividendo dati e interagendo per offrire servizi distribuiti ed autonomi. Basti pensare al semplice SSID di un Accesspoint WiFi, che si annuncia rendendosi visibile ai dispositivi in grado di connettersi, e diffondendo il proprio nome/ID del Router/Accesspoint; o ad una semplice stampante che si rende disponibile all'interno di una rete locale, identificandosi con il codice del Modello e diffondendo varie altre informazioni. 

La necessit� di rendere pi� autonoma possibile la comunicazione tra dispositivi e applicazioni, ha portato allo sviluppo di numerose tecniche e protocolli cos� detti di \textit{Autoconfigurazione}, che permettono un setup autonomo del/dei dispositivo/i all'interno della rete locale, e quindi non necessitando di una configurazione \textquotedblleft manuale\textquotedblright. 

Tutte queste interazioni sono rese possibili grazie la diffusione di informazioni, pi� o meno confidenziali, a seconda del protocollo utilizzato, e spesso compromettendo la privacy dell'utente, possessore del dispositivo o utilizzatore della specifica applicazione. 

Inoltre possono essere presenti e ammesse all'interno della rete, tecnologie/protocolli che vanno a minare direttamente la sicurezza dell'intera rete, rendendo possibile modifiche nella configurazione di router o device di rete direttamente da remoto, come UPnP: il quale permette di aprire porte all'interno del router locale senza la necessit� di autenticazione o permessi specifici.
\newpage
Indubbiamente con tutte queste tecnologie si sono semplificate, se non addirittura rese completamente automatizzate, molte procedure di configurazione e interconnessione, rendendo accessibile a chiunque l'utilizzo di tali strumenti. Ma a quale prezzo? L'utilizzatore � a conoscenza di quali sono le informazioni scambiate all'interno della propria rete locale, e quali dati rende disponibile ad un eventuale ospite esterno/intruso nella propria Home-Network?  

\section[Traffico LAN Prima]{Traffico LAN Prima}
Fino a qualche anno fa, all'interno delle nostre reti locali private, la quantit� di traffico interno che vi transitava era pressoch� nulla, dato che le uniche periferiche che avevano accesso alle rete erano i PC, e l'interazione fra di loro e le applicazioni era minima.

I primi a tentare un approccio di \textquotedblleft Autoconfigurazione\textquotedblright e di \textit{interconnessione automatica} furono gli sviluppatori Apple con il loro AppleTalk: in grado di mettere in comunicazione un gruppo di Macs all'interno di una rete locale LAN senza bisogno dell'intervento di alcun esperto, senza la necessit� di alcun setup o di una struttura centrale che coordinasse o offrisse servizi per le periferiche, come un server DHCP o di un server DNS. Similmente, in seguito furono sviluppati NetBIOS e IPX, offrendo la medesima possibilit� di interconnettere dispositivi che implementassero i suddetti protocolli.

Con l'avvento dello standard(tutt'oggi ancora NON definitivo) denominato \textbf{\textit{Zeroconf}}, nato dall'idea di AppleTalk, si sono susseguite numerose implementazioni e copiosi utilizzi del concetto di \textit{autoconfigurazione} e 0 intervento esterno/strutture centrali per la configurazione e il coordinamento fra applicazioni/dispositivi. I primi a trarne vantaggio e trovarne subito un pratico utilizzo furono i costruttori di stampanti e, in generale, di dispositivi utilizzati in ufficio, non avendo avuto fino a quel momento la possibilit� di includere interfacce utente per configurare manualmente le macchine, e quindi rendendo impossibile un agevole utilizzo di tali apparecchi all'interno della rete Aziendale/Domestica.

\newpage
\section[Traffico LAN Oggi]{Cosa transita oggi all'interno \\delle NOSTRE Reti?}
La quantit� di informazioni che transita oggi all'interno della rete locale � veramente vasta, rendendo possibile l'utilizzo di dispositivi e servizi da essi offerti anche ad utenti non specializzati, completamente ignari di come sia resa possibile l'interazione; di contro, tutti questi dati, non solo riportano molte informazioni personali su dispositivi ed utenti che li utilizzano, ma inoltre per carpirle non � necessario compiere azioni specifiche, come introdursi all'interno del dispositivo, � sufficiente essere collegati alla stessa rete locale e recuperare tali informazioni dai pacchetti che vengono liberamente distribuiti all'interno della stessa, sia che il dispositivo sia realmente interessato che non. Questa situazione pone un'eventuale intruso/ospite nella rete, che � interessato a scoprirne la topologlia, in una condizione ottimale, limitandosi ad \textit{ascoltare} le informazioni che gli vengono fornite degli altri dispositivi, senza intraprendere azioni di alcuna sorta nei confronti degli altri dispositivi, e quindi rendendone anche difficile l'individuazione. 

\section{Obiettivo}
Lo scopo di questo Lavoro di Tesi � mostrare le vulnerabilit� delle reti locali in termini di dati sensibili e privati, cercando di acquisire il maggior numero di informazioni possibili riguardo i nodi della rete, in modo completamente passivo, e identificando quali dispositivi sono connessi attualmente alla Rete Locale a cui abbiamo accesso. Questo mette in evidenza quante e quali informazioni vengono scambiate all'interno della rete, rendendo consapevoli gli utilizzatori di tale rete, di quali saranno le informazioni private che verranno diffuse tramite i loro dispositivi ad essa collegati. Grazie a tale consapevolezza, chi � addetto alla gestione e progettazione della rete pu� decidere eventualmente di separare il traffico in sotto-reti isolate, in modo tale da arginare eventuali diffusioni di informazioni sensibili, pur mantenendo e usufruendo di tutti i vantaggi che una comunicazione Broadcast/Multicast fra dispositivi in una rete locale comporta: per esempio auto-configurazione e scambio rapido di dati all'interno della rete locale, relegandolo nella propria LAN.

Al fine di raggiungere tale scopo, � stato approntato uno studio sulla metodologia di raccolta di tali informazioni, e la loro organizzazione ed elaborazione, identificando la natura dei dispositivi che popolano una generica rete locale, i servizi da loro offerti, e in alcuni casi, i rapporti/connessioni che hanno fra loro.

Come risultato di tale studio, � stato implementato uno strumento per l'analisi automatica di una rete locale, in grado di fornire informazioni pi� o meno dettagliate riguardo la topologia della rete, alla quale si ha libero accesso, e quindi identificando tutti quei dispositivi che annunciano e offrono servizi al suo interno: Dispositivi Mobili, Stampanti, Workstation di vario genere e Media-devices.

%%%%%%%%%%%%%%%%%%%%%%%%%%%%%%%%%%%%%%%%%%%%%%%%%%%%%%%%%%%

\section{Struttura della tesi}

Riassumere sommariamente ogni capitolo.

\begin{description}
	\item[Capitolo \ref{ref:Lavoro}] Panoramica dello studio approntato e del tool sviluppato per questo lavoro di Tesi.

	\item[Capitolo \ref{ref:Arte}] Quali sono gli altri studi a riguardo.

	\item[Capitolo \ref{ref:Implementazione}] Descrizione dettagliata degli strumenti usati e sull'implementazione.

	\item[Capitolo \ref{ref:Conclusioni}] Conclusioni tratte da questo studio e spunti per lavori futuri.

	\item[Appendice \ref{ref:appA}] Scrivere il riassunto.

	\item[Appendice \ref{ref:appB}] Scrivere il riassunto.
\end{description}

%%%%%%%%%%%%%%%%%%%%%%%%%%%%%%%%%%%%%%%%%%%%%%%%%%%%%%%%%%%

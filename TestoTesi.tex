\documentclass[12pt,a4paper,openright]{book}
\usepackage[italian]{babel}
\usepackage[T1]{fontenc}
\usepackage[latin1]{inputenc}
\usepackage{amsmath}
\usepackage{amsfonts}
\usepackage{amssymb}
\usepackage{graphicx}
\author{Edoardo Maione}
\title{Qui va il Titolo}
\makeindex
\begin{document}
	\maketitle
	\begin{minipage}[c][\textheight][c]{\textwidth}
		\pagestyle{empty}

		\textbf{\Large Abstract} \\
		
		Nelle Reti locali moderne, sia Domestiche che Aziendali, sta crescendo il numero di dispositivi ad alta capacit� di interazione con gli altri all'interno della stessa rete. A partire da stampanti di rete, proseguendo con i dispositivi mobili come Smartphone, Tablet, Wearable, ecc\dots \space che hanno avuto un aumento di diffusione esponenziale nell'ultimo decennio, fino ad arrivare agli elettrodomestici e alla domotica, anch'essa in crescita. In questo elaborato si vuole studiare e conoscere quali sono le informazioni che vengono scambiate tra questi dispositivi all'interno della rete locale per permettere un'interazione efficiente, ma soprattutto una ridotta, se non nulla, necessit� di intervento \textquotedblleft specializzato\textquotedblright \space per quanto riguarda la configurazione di tali apparecchi.
		
	\end{minipage}
	\tableofcontents
	\begin{minipage}[c][\textheight][c]{\textwidth}
		\index{\makeindex}
	\end{minipage}
	\newpage
	\chapter{Introduzione}
	Dispositivi e Applicazioni sono sempre pi� interconnessi tra loro, comunicando e scambiando informazioni, condividendo dati e interagendo per offrire servizi distribuiti ed autonomi. Basti pensare al semplice SSID di un Accesspoint WiFi, che si annuncia rendendosi visibile ai dispositivi in grado di connettersi, e diffondendo il proprio nome/ID del Router/Accesspoint; o ad una semplice stampante che si rende disponibile all'interno di una rete locale, identificandosi con il codice del Modello e diffondendo varie altre informazioni. 
	
	La necessit� di rendere pi� autonoma possibile la comunicazione tra dispositivi e applicazioni, ha portato allo sviluppo di numerose tecniche e protocolli cos� detti di \textit{Autoconfigurazione}, che permettono un setup autonomo del/dei dispositivo/i all'interno della rete locale, e quindi non necessitando di una configurazione \textquotedblleft manuale\textquotedblright. 
	
	Tutte queste interazioni sono rese possibili grazie la diffusione di informazioni, pi� o meno confidenziali, a seconda del protocollo utilizzato, e spesso compromettendo la privacy dell'utente, possessore del dispositivo o utilizzatore della specifica applicazione. 
	
	Inoltre possono essere presenti e ammesse all'interno della rete, tecnologie/protocolli che vanno a minare direttamente la sicurezza dell'intera rete, rendendo possibile modifiche nella configurazione di router o device di rete direttamente da remoto, come UPnP: il quale permette di aprire porte all'interno del router locale senza la necessit� di autenticazione o permessi specifici.
	\newpage
	Indubbiamente con tutte queste tecnologie si sono semplificate, se non addirittura rese completamente automatizzate, molte procedure di configurazione e interconnessione, rendendo accessibile a chiunque l'utilizzo di tali strumenti. Ma a quale prezzo? L'utilizzatore � a conoscenza di quali sono le informazioni scambiate all'interno della propria rete locale, e quali dati rende disponibile ad un eventuale ospite esterno/intruso nella propria Home-Network? 
	
	\section[Traffico LAN Prima]{Traffico LAN Prima}
	Fino a qualche anno fa, all'interno delle nostre reti locali private, la quantit� di traffico interno che vi transitava era pressoch� nulla, dato che le uniche periferiche che avevano accesso alle rete erano i PC, e l'interazione fra di loro e le applicazioni era minima.
	
	I primi a tentare un approccio di \textquotedblleft Autoconfigurazione\textquotedblright e di \textit{interconnessione automatica} furono gli sviluppatori Apple con il loro AppleTalk: in grado di mettere in comunicazione un gruppo di Macs all'interno di una rete locale LAN senza bisogno dell'intervento di alcun esperto, senza la necessit� di alcun setup di una struttura centrale che coordini o offra servizi per le periferiche, come un server DHCP o di un server DNS. Similmente, in seguito furono sviluppati NetBIOS e IPX, offrendo la medesima possibilit� di interconnettere dispositivi che implementassero i suddetti protocolli.
	
	Con l'avvento dello standard(tutt'oggi ancora NON definitivo) denominato \textbf{\textit{Zeroconf}}, nato dall'idea di AppleTalk, si sono susseguite numerose implementazioni e copiosi utilizzi del concetto di \textit{autoconfigurazione} e 0 intervento esterno/strutture centrali per la configurazione e il coordinamento fra applicazioni/dispositivi. I primi a trarne vantaggio e trovarne subito un pratico utilizzo furono i costruttori di stampanti e, in generale, di dispositivi utilizzati in ufficio, non avendo avuto fino a quel momento la possibilit� di includere interfacce utente per configurare manualmente le macchine, e quindi rendendo impossibile un agevole utilizzo di tali apparecchi all'interno della rete Aziendale/Domestica.
	
	\newpage
	\section[Traffico LAN Oggi]{Cosa transita all'interno delle NOSTRE Reti?}
	\section[Il mio Lavoro]{analisi mDNS e DB-lsp-DISC}
	Qui descrivo in breve che ho fatto: ... 
	
	\chapter[Broadcast/Multicast LAN]{Accenni a protocolli Broadcast/Multicast nelle reti Locali}
	\section[Zeroconf]{Zeroconf title}
	\subsection[AppleTalk]{AppleTalk title}
	\subsection[Implementazioni]{Implementazioni}
	\subsubsection[Bonjour]{Bonjour title}
	\subsubsection[Avahi]{Avahi title}
	
	\section[UPnP]{UPnP title} 
	\subsection[ssdp]{Simple Service Discovery Protocol}
	
	\section[mDNS]{mDNS title}
	\subsection[mDNS-SD]{mDNS-SD title}
	\subsubsection[Service1]{ Service1 title}
	\subsubsection[Service2]{ Service2 title}
	\subsubsection[Service...]{ Service... title}
	
	\section[DB-LSP-DISC]{DropBox LSP DISCovery title}
	
	\chapter[Mio Lavoro]{Il mio lavoro}
	Lo scopo di questo lavoro � mostrare le vulnerabilit� delle reti locali in termini di dati sensibili e privati, cercando di acquisire il maggior numero di informazioni possibili riguardo i nodi della rete, in modo completamente passivo, e identificando quali dispositivi sono connessi attualmente alla Rete Locale a cui abbiamo accesso. Questo mette in evidenza quante e quali informazioni vengono scambiate all'interno della rete, rendendo consapevoli gli utilizzatori di tale rete, di quali saranno le informazioni private che verranno diffuse tramite i loro dispositivi ad essa collegati. Grazie a tale consapevolezza, chi � addetto alla gestione della rete pu� decidere eventualmente di separare il traffico in sotto-reti isolate, in modo tale da arginare eventuali diffusioni di informazioni sensibili, pur mantenendo e usufruendo di tutti i vantaggi che una comunicazione Broadcast/Multicast fra dispositivi in una rete locale comporta: per esempio auto-configurazione e scambio rapido di informazioni all'interno della rete locale, limitandole a quest'ultima ... ??? .   
	
	
	\section{Strumenti usati}
	Il mio lavoro � stato reso possibile grazie ad un wrapper di tshark scritto in python chiamato \textit{pyshark}, il quale fornisce delle funzioni d'interfaccia che permettono di catturare/leggere file di cattura ???  , ed accedere ai campi che compongono i vari pacchetti, estraendo informazioni utili.
	
	\textit{Pyshark} � un wrapper per tshark, reperibile sulla piattaforma GitHub ... .
	Non � propriamente un dissector, come molti altri, ma si limita a usare la funzionalit� di tshark di esportare XMLs per usare il suo parsing. 
	
	\textit{WireShark} (\textit{TShark})� un software gratuito che permette di catturare il traffico della rete che transita sulla propria scheda di rete, senza la necessit� porsi in punti di snodo \textquotedblleft centrali\textquotedblright, come Router o Accesspoint. Questo limita molto il traffico dati catturabile, ma permette comunque di ottenere informazioni su tutti i pacchetti inviati a altri dispositivi in Broadcast/Multicast, che per i nostri scopi � sufficiente.
	Mette in condizioni di non rendere visibile al resto della rete che si sta analizzando del traffico dati ... ??? .
	\textit{Tshark}, nello specifico, � una utility \textquotedblleft a linea di comando\textquotedblright\space di Wireshark, che utilizza quindi il core del programma principale, offrendo le medesime funzionalit�
	
	\section[mDNS]{mDNS Response dissection \& study title}
	\section[DB-LSP-DISC]{Dropbox namespaces title}
	
	\chapter[Conclusioni]{Conclusioni}
	\section{Lavori Futuri}
	
\end{document}